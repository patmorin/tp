\documentclass{patmorin}
\listfiles
\usepackage{pat}
\usepackage{paralist}
\usepackage{dsfont}  % for \mathds{A}
\usepackage[utf8x]{inputenc}
\usepackage{skull}
\usepackage{paralist}
\usepackage{graphicx}
\usepackage[noend]{algorithmic}
\usepackage{bbm}  % needed for \mathbbm{1}
\usepackage{listings}

\usepackage[normalem]{ulem}
\usepackage{cancel}
%\usepackage{enumitem}

\usepackage{todonotes}

% etoolbox allows for robust commands that don't need \protect, e.g.
% \newrobustcmd{\onesub}{\mathord{\includegraphics{figs/one-sub}}}
% \subsection{Approximate Voronoi Diagrams in $G^{\onesub}_k$}
\usepackage{etoolbox}

\usepackage[longnamesfirst,numbers,sort&compress]{natbib}

\usepackage[mathlines]{lineno}
\setlength{\linenumbersep}{2em}
% \linenumbers
% \rightlinenumbers
% \linenumbers
\newcommand*\patchAmsMathEnvironmentForLineno[1]{%
 \expandafter\let\csname old#1\expandafter\endcsname\csname #1\endcsname
 \expandafter\let\csname oldend#1\expandafter\endcsname\csname end#1\endcsname
 \renewenvironment{#1}%
    {\linenomath\csname old#1\endcsname}%
    {\csname oldend#1\endcsname\endlinenomath}}%
\newcommand*\patchBothAmsMathEnvironmentsForLineno[1]{%
 \patchAmsMathEnvironmentForLineno{#1}%
 \patchAmsMathEnvironmentForLineno{#1*}}%
\AtBeginDocument{%
\patchBothAmsMathEnvironmentsForLineno{equation}%
\patchBothAmsMathEnvironmentsForLineno{align}%
\patchBothAmsMathEnvironmentsForLineno{flalign}%
\patchBothAmsMathEnvironmentsForLineno{alignat}%
\patchBothAmsMathEnvironmentsForLineno{gather}%
\patchBothAmsMathEnvironmentsForLineno{multline}%
}

\newcommand{\vol}[1]{\boxplus_{#1}}


\DeclareMathOperator{\interior}{Int}

% Taken from
% https://tex.stackexchange.com/questions/42726/align-but-show-one-equation-number-at-the-end
\newcommand\numberthis{\addtocounter{equation}{1}\tag{\theequation}}

\definecolor{brightmaroon}{rgb}{0.76, 0.13, 0.28}
\definecolor{linkblue}{rgb}{0, 0.337, 0.227}
\newcommand{\defin}[1]{\emph{\color{brightmaroon}#1}}
\setlength{\parskip}{1ex}

% Document-specific commands and math operators
\DeclareMathOperator{\tw}{tw}
\DeclareMathOperator{\td}{td}
\DeclareMathOperator{\chicen}{\chi_{\mathrm{cen}}}
\DeclareMathOperator{\chilin}{\chi_{\mathrm{lin}}}
\DeclareMathOperator{\dist}{dist}
\DeclareMathOperator{\vor}{Vor}

\DeclareMathOperator{\binomial}{binomial}

\newrobustcmd{\onesub}{\mathord{\includegraphics{figs/one-sub}}}
\newrobustcmd{\leftup}{\mathord{\includegraphics{figs/left-up}}}

\title{\MakeUppercase{Bad News for Product Structure of Bounded-Degree Graphs}\thanks{This research was partly funded by NSERC.}}
% \author{Prosenjit~Bose, Vida~Dujmović, Hussein~Houdrouge, Mehrnoosh~Javarsineh, and Pat~Morin}
\author{TBD}

\DeclareMathOperator{\VE}{\mathit{VE}}

\date{}


\begin{document}

\maketitle
\renewcommand{\E}{\mathbb{E}}
\renewcommand{\Pr}{\mathbb{P}}

\begin{abstract}
  We give an example of a family $\mathcal{G}$ of planar graphs of maximum degree $5$ such that, for any tree $T$, any path $P$, and any clique $K$, there exists a member of $\mathcal{G}$ that is not isomorphic to any subgraph of $T\boxtimes P\boxtimes K$.
\end{abstract}

\section{Introduction}

For two graphs $G_1$ and $G_2$ we use the notation $G_1=G_2$ to mean that $G_1$ and $G_2$ are isomorphic and the notation $G_1\subseteq G_2$ to mean that $G_1$ is isomorphic to some subgraph of $G_2$.  Note that, with this definition, the strong product of graphs is commutative, i.e., $G_1\boxtimes G_2=G_2\boxtimes G_1$.

We say that a graph $G$ is a \emph{max-degree-$\Delta$ graph} if $\Delta \le \max\{\deg_G(v):v\in V(G)\}$.

For any positive integer $p$, $K_p$ denotes the complete graph on $p$ vertices.  Note that for any positive integers $p$ and $q$, $K_{p}\boxtimes K_{q}=K_{pq}$.

\begin{thm}\label{treewidth_1_bounded_degree}
  For every integer $n\ge 1$, there exists an $n$-vertex max-degree-$5$ planar graph $G$ such that, for any treewidth-$t$ max-degree-$\Delta$ graph $H$, $G\subseteq T\boxtimes P\boxtimes K_r$ implies that $r\ge n^{1-O(\log (t\Delta)/\log\log n)}$.
\end{thm}

To obtain a short proof of \cref{treewidth_1_bounded_degree}, we make use of the following theorem, due to \citet{ding.oporowski:some} (see also \citet{wood:on}).\footnote{References \cite{ding.oporowski:some,wood:on} state \cref{tree_partition_theorem} in terms of tree partition width. It is straightforward to verify, using only the definitions of tree partition width and strong product that the two statements are equivalent.}

\begin{thm}[\citet{ding.oporowski:some}]\label{tree_partition_theorem}
  There exists a constants $a,b>0$ such that, for any treewidth-$t$ max-degree-$\Delta$ graph $G$, there exists a max-degree-$at\Delta$ tree $T$ such that $G\subseteq T\boxtimes K_{bt\Delta}$.\todo{\cite{ding.oporowski:some,wood:on} don't mention the degree of $T$. Reference?}
\end{thm}


\begin{cor}
  There exists a constants $a,b>0$ such that, for any treewidth-$t$ maximum-degree-$\Delta$ graph of $H$, any path $P$, and any $r\ge 1$, there exists a max-degree-$at\Delta$ tree $T$ such that
  $H\boxtimes P\boxtimes K_r = T\boxtimes P\boxtimes K_{rbt\Delta}$.
\end{cor}

\begin{proof}
  Apply \cref{tree_partition_theorem} to $H$ to obtain a maximum-degree-$a\Delta$ tree $T$ such that $H\subseteq T\boxtimes K_{bt\Delta}$.  Then $H\boxtimes P\boxtimes K_{c} = T\boxtimes K_{bt\Delta}\boxtimes P\boxtimes K_c=T\boxtimes P\boxtimes K_{cbt\Delta}$.
\end{proof}

\begin{lem}\label{easy_version}
  There exists an $n$-vertex max-degree-$5$ planar graph $G$ such that, for any max-degree-$\Delta$ tree $T$, $G\subseteq T\boxtimes P\boxtimes K_r$ implies that $r\ge n^{1-O(\log\Delta/\log\log n)}$.
\end{lem}

Before proving \cref{easy_version}, we show how it implies \cref{treewidth_1_bounded_degree}.

\begin{proof}[Proof of \cref{treewidth_1_bounded_degree}]
  Let $G$ be the graph whose existence is the subject of \cref{easy_version}.
  Let $H$ be a treewidth-$t$ max-degree-$\Delta$ graph such that $G\subseteq H\boxtimes P\boxtimes K_r$.  Then, by \cref{easy_version}, there exists a max-degree-$at\Delta$ tree $T$ such that $G\subseteq T\boxtimes P\boxtimes K_{brt\Delta}$.  Then $brt\Delta = n^{1-O(\log\Delta/\log\log n)}$, so $r\ge n^{1-O(\log (t\Delta)/\log\log n)}$.
\end{proof}

\begin{proof}[Proof of \cref{easy_version}]
  For simplicity, let $n=2^{h+1}-1$, and let $G_0$ be the complete ordered binary tree of height $H$ (so $G_0$ has exactly $n$ vertices).  Define $G$ to be the supergraph of $G_0$ obtained by adding the edges of a path $P_i$ through all the nodes of depth $h-i$ in $G_0$, for each $i\in\{0,\ldots,h-1\}$.\todo{Show figure}.  Suppose $G\subseteq T\boxtimes P\boxtimes K_r$.  Fix an injection $\rho:V(G)\to V(T\boxtimes P\boxtimes K_r)$ that witnesses the fact that $G$ is a subgraph of $T\boxtimes P\boxtimes K_r$. Consider the path $P_0$ 
\end{proof}



\bibliographystyle{plainurlnat}
\bibliography{tp}

\end{document}
