\documentclass{patmorin}
\listfiles
\usepackage{pat}
\usepackage{paralist}
\usepackage{dsfont}  % for \mathds{A}
\usepackage[utf8x]{inputenc}
\usepackage{skull}
\usepackage{paralist}
\usepackage{graphicx}
\usepackage[noend]{algorithmic}
\usepackage{bbm}  % needed for \mathbbm{1}
\usepackage{listings}

\usepackage[normalem]{ulem}
\usepackage{cancel}
%\usepackage{enumitem}

\usepackage{todonotes}

% etoolbox allows for robust commands that don't need \protect, e.g.
% \newrobustcmd{\onesub}{\mathord{\includegraphics{figs/one-sub}}}
% \subsection{Approximate Voronoi Diagrams in $G^{\onesub}_k$}
\usepackage{etoolbox}

\usepackage[longnamesfirst,numbers,sort&compress]{natbib}

\usepackage[mathlines]{lineno}
\setlength{\linenumbersep}{2em}
% \linenumbers
% \rightlinenumbers
% \linenumbers
\newcommand*\patchAmsMathEnvironmentForLineno[1]{%
 \expandafter\let\csname old#1\expandafter\endcsname\csname #1\endcsname
 \expandafter\let\csname oldend#1\expandafter\endcsname\csname end#1\endcsname
 \renewenvironment{#1}%
    {\linenomath\csname old#1\endcsname}%
    {\csname oldend#1\endcsname\endlinenomath}}%
\newcommand*\patchBothAmsMathEnvironmentsForLineno[1]{%
 \patchAmsMathEnvironmentForLineno{#1}%
 \patchAmsMathEnvironmentForLineno{#1*}}%
\AtBeginDocument{%
\patchBothAmsMathEnvironmentsForLineno{equation}%
\patchBothAmsMathEnvironmentsForLineno{align}%
\patchBothAmsMathEnvironmentsForLineno{flalign}%
\patchBothAmsMathEnvironmentsForLineno{alignat}%
\patchBothAmsMathEnvironmentsForLineno{gather}%
\patchBothAmsMathEnvironmentsForLineno{multline}%
}

\newcommand{\vol}[1]{\boxplus_{#1}}


\DeclareMathOperator{\interior}{Int}

% Taken from
% https://tex.stackexchange.com/questions/42726/align-but-show-one-equation-number-at-the-end
\newcommand\numberthis{\addtocounter{equation}{1}\tag{\theequation}}

\definecolor{brightmaroon}{rgb}{0.76, 0.13, 0.28}
\definecolor{linkblue}{rgb}{0, 0.337, 0.227}
\newcommand{\defin}[1]{\emph{\color{brightmaroon}#1}}
\setlength{\parskip}{1ex}

% Document-specific commands and math operators
\DeclareMathOperator{\tw}{tw}
\DeclareMathOperator{\td}{td}
\DeclareMathOperator{\chicen}{\chi_{\mathrm{cen}}}
\DeclareMathOperator{\chilin}{\chi_{\mathrm{lin}}}
\DeclareMathOperator{\dist}{dist}
\DeclareMathOperator{\diam}{diam}
\DeclareMathOperator{\vor}{Vor}

\DeclareMathOperator{\binomial}{binomial}

\newrobustcmd{\onesub}{\mathord{\includegraphics{figs/one-sub}}}
\newrobustcmd{\leftup}{\mathord{\includegraphics{figs/left-up}}}

\title{\MakeUppercase{Bad News for Product Structure of Bounded-Degree Graphs}}
% \author{Prosenjit~Bose, Vida~Dujmović, Hussein~Houdrouge, Mehrnoosh~Javarsineh, and Pat~Morin}
\author{%
  Prosenjit Bose\thanks{School of Computer Science, Carleton University, Ottawa, Canada}\, \thanks{Research partly supported by NSERC.} \quad
  Gwenaël Joret\thanks{Affilation/funding} \quad
  Vida Dujmović\thanks{Department of Computer Science and Electrical Engineering, University of Ottawa, Ottawa, Canada.}\,\, \footnotemark[2] \quad
  Piotr Micek\thanks{Affilation/funding}\quad
  Pat Morin\footnotemark[1]\, \footnotemark[2] \quad
  
  David R. Wood\thanks{School of Mathematics, Monash University, Melbourne, Australia (\texttt{david.wood@monash.edu}). Research supported by the Australian Research Council.
}}

\DeclareMathOperator{\VE}{\mathit{VE}}

\date{}


\begin{document}

\maketitle
\renewcommand{\E}{\mathbb{E}}
\renewcommand{\Pr}{\mathbb{P}}

\begin{abstract}
  We give an example of a family $\mathcal{G}$ of planar graphs of maximum degree $5$ such that, if an $n$-vertex member of $\mathcal{G}$ is a subgraph of the product of a graph $H$, a path $P$ and a clique $K$ where $H$ has treewidth $t$ and maximum degree $\Delta$ and $K$ has order $r$, then $\log(t\Delta r) \in \Omega(\log n)$.
\end{abstract}

\section{Introduction}

Recently, product structure theorems and the related notion of row treewidth have been a key tool in resolving a number of longstanding open problems on planar graphs.  Roughly, a \defin{product structure theorem} for a graph family $\mathcal{G}$ states that there exists integers $t$ and $r$ such that, for every $G\in\mathcal{G}$ there is a graph $H$ of treewidth at most $t$ and a path $P$ such that $G$ is isomorphic to a subgraph of the strong product of $H$, $P$, and a clique $K$ of order $r$.\footnote{A \defin{tree decomposition} of a graph $H$ is a sequence $\mathcal{T}:=(B_x:x\in V(T))$ of subsets of $V(H)$ indexed by the nodes of some tree $T$ such that
\begin{inparaenum}[(i)]
  \item for each $v\in V(H)$, the induced subgraph $T[x\in V(T):v\in B_x]$ is connected; and
  \item for each edge $vw\in E(H)$, there exists some $x\in V(T)$ with $\{v,w\}\subseteq B_x$.
\end{inparaenum}
The \defin{width} of a tree decomposition of $\mathcal{T}$ is $\max\{|B_x|:x\in V(T)\}-1$. The \defin{treewidth} of $H$ is the minimum width of any tree decomposition of $H$.}\footnote{The \defin{strong product} $G_1\boxtimes G_2$ of two graphs $G_1$ and $G_2$ is the graph with vertex set $V(G_1\boxtimes G_2):=V(G_1)\times V(G_2)$ and that includes the edge with endpoints $(v,x)$ and $(w,y)$ if and only if
\begin{inparaenum}[(i)]
  \item $vw\in E(G_1)$ and $x=y$;
  \item $v=w$ and $xy\in E(G_2)$; or
  \item $vw\in E(G_1)$ and $xy\in E(G_2)$.
\end{inparaenum}
}
This is typically written as $G\subseteq H\boxtimes P\boxtimes K$, where the notation $G_1\subseteq G_2$ is used to mean that $G_1$ is isomorphic to some subgraph of $G_2$.

In applications of product structure theorems, it is often helpful if, in addition to having treewidth $t$, the graph $H$ has additional useful properties, possibly inherited from $G$.  For example, one very useful version of the planar product structure theorem states that for every planar graph $G$ there exists $H$ and $P$ with $G\subseteq H\boxtimes P\boxtimes K_3$ where the treewidth of $H$ is at most $3$ and $H$ is also \emph{planar}.  The planarity of $H$ in this result has been leveraged to obtain better constants and even asymptotic improvements for a number of graph colouring and layout problems.

One question that arises frequently (and which the authors have been asked repeatedly) is the following:
\begin{quote}
  Let $\mathcal{G}_\Delta$ be the family of planar graphs of maximum degree $\Delta$.  Does $\mathcal{G}_\Delta$ have a product structure theorem in which (in addition to having treewidth $t$) $H$ has maximum degree that is bounded by a function of $\Delta$?
\end{quote}
In this note we show that, unfortunately, the answer to this question is no, by proving the following:
% , even with $\Delta=5$.  We give a family of max-degree-$5$ planar graphs such that if an $n$-vertex member of this family is a subgraph of $H\boxtimes P\boxtimes K$ where $H$ has treewidth $t$, $H$ has maximum degree $\Delta$, and $K$ is a clique of order $r$, then $\log(r\Delta t)\in\Omega(\log\log n)$. In particular, we prove the following theorem:

\begin{thm}\label{treewidth_1_bounded_degree}
  For infinitely many integers $n\ge 1$, there exists an $n$-vertex max-degree-$5$ planar graph $G$ such that, for any treewidth-$t$ max-degree-$\Delta$ graph $H$, any path $P$, and any integer $r$,  if $G\subseteq H\boxtimes P\boxtimes K_r$ then $\log(t\Delta r)\in\Omega(\log n)$.
\end{thm}

\section{Proof of \cref{treewidth_1_bounded_degree}}

In this paper, all graphs are finite, undirected, and simple.  For two graphs $G_1$ and $G_2$ we use the notation $G_1=G_2$ to mean that $G_1$ and $G_2$ are isomorphic and the notation $G_1\subseteq G_2$ to mean that $G_1$ is isomorphic to some subgraph of $G_2$.  Note that, with this definition, the strong product of graphs is commutative, i.e., $G_1\boxtimes G_2=G_2\boxtimes G_1$.  Furthermore for any positive integers $p$ and $q$, $K_{p}\boxtimes K_{q}=K_{pq}$.

We say that a graph $G$ is a \defin{max-degree-$\Delta$ graph} if it contains no vertex of degree greater than $\Delta$.  For a connected graph $G$ and two vertices $v,w\in V(G)$, $\dist_G(v,w)$ denotes the length of a shortest path in $G$ with endpoints $v$ and $w$.  The \defin{diameter} of $G$ is denoted $\diam(G):=\max\{\dist_G(v,w):v,w\in V(G)\}$.

To obtain a short proof of \cref{treewidth_1_bounded_degree}, we make use of the following theorem, due to \citet{ding.oporowski:some} (see also \citet{wood:on}).\footnote{References \cite{ding.oporowski:some,wood:on} state \cref{tree_partition_theorem} in terms of tree partition width. It is straightforward to verify, using only the definitions of tree partition width and strong product that the two statements are equivalent.}

\begin{thm}[\citet{ding.oporowski:some}]\label{tree_partition_theorem}
  There exists a constant $c>0$ such that, for any treewidth-$t$ max-degree-$\Delta$ graph $G$, there exists a max-degree-$ct\Delta^2$ tree $T$ such that $G\subseteq T\boxtimes K_{ct\Delta}$.\todo{\cite{ding.oporowski:some,wood:on} don't mention the degree of $T$, but $ct\Delta^2$ is obvious, easy, and good enough. Reference?}
\end{thm}

\cref{tree_partition_theorem} allows us to prove \cref{treewidth_1_bounded_degree} by focusing on the case in which $H$ has treewidth $1$, because of the following corollary:

\begin{cor}\label{tree_partition_corollary}
  There exists a constant $c>0$ such that, for any treewidth-$t$ maximum-degree-$\Delta$ graph $H$, any path $P$, and any $r\ge 1$, there exists a max-degree-$ct\Delta^2$ tree $T$ such that $H\boxtimes P\boxtimes K_r \subseteq T\boxtimes P\boxtimes K_{crt\Delta}$.
\end{cor}

\begin{proof}
  Apply \cref{tree_partition_theorem} to $H$ to obtain a maximum-degree-$ct\Delta^2$ tree $T$ such that $H\subseteq T\boxtimes K_{ct\Delta}$.  Then $H\boxtimes P\boxtimes K_{r} \subseteq T\boxtimes K_{ct\Delta}\boxtimes P\boxtimes K_r=T\boxtimes P\boxtimes K_{crt\Delta}$.
\end{proof}

Thus, we establish \cref{treewidth_1_bounded_degree} using the following lemma.

\begin{lem}\label{easy_version}
  For infinitely many positive integers $n$, there exists an $n$-vertex max-degree-$5$ planar graph $G$ such that, for any max-degree-$\Delta$ tree $T$, if $G\subseteq T\boxtimes P\boxtimes K_r$ then $\log (\Delta r)\ge (1-o(1))\log n$.
\end{lem}

Before proving \cref{easy_version}, we show how it implies \cref{treewidth_1_bounded_degree}.

\begin{proof}[Proof of \cref{treewidth_1_bounded_degree}]
  Let $G$ be an $n$-vertex graph whose existence is the subject of \cref{easy_version}.  Let $H$ be a treewidth-$t$ max-degree-$\Delta$ graph such that $G\subseteq H\boxtimes P\boxtimes K_r$.  Then, by \cref{easy_version}, there exists a max-degree-$ct\Delta^2$ tree $T$ such that $G\subseteq T\boxtimes P\boxtimes K_{crt\Delta}$.  Since $G$ comes from \cref{easy_version}, $\log(c^2rt^2\Delta^3)\ge (1-o(1))\log n$. \cref{treewidth_1_bounded_degree} now, since $\log(t\Delta r)\in\Omega(\log crt^2\Delta^3)$.
\end{proof}

\begin{proof}[Proof of \cref{easy_version}]
  Let $n:=2^{h+1}-1$ for some large positive integer $h$, and let $G_0$ be a complete binary tree of height $h$ (so $G_0$ has exactly $n$ vertices and $2^h$ leaves).  Define $G$ to be the supergraph of $G_0$ obtained by adding, for each $i\in\{0,\ldots,h\}$, the edges of a path $P_i$ that contains all $2^{h-i}$ nodes of depth $h-i$ in $G_0$.  Observe that $\diam(G)\le\diam(G_0)=2h$.

  Now suppose $G\subseteq T\boxtimes P\boxtimes K_r$ and fix an injective homomorphism $\rho:V(G)\to V(T\boxtimes P\boxtimes K_r)$.  For two vertices $v$ and $w$ of $G$ with $\rho(v):=(v_1,v_2,v_3)$ and $\rho(w):=(w_1,w_2,w_3)$ we call $v_1$, $v_2$, and $v_3$ the \defin{projections} of $v$ onto $T$, $P$, and $K_r$, respectively.  For simplicity, we define $\dist_T(v,w):=\dist_T(v_1,w_1)$ and $\dist_P(v,w):=\dist_P(v_2,w_2)$.  Then, by the definition of strong product, $\dist_G(v,w)\ge \max\{\dist_T(v,w),\dist_P(v,w)\}$.  Since $\diam(G)\le 2h$, this allows us to assume that $\diam(P)\le 2h$ and that $\diam(T)\le 2h$.  The former inequality implies that $|V(P)|\le 2h+1$.  However, the latter inequality only implies that $|V(T)|\in O(n^2)$, which is insufficient for our purposes.

  Our strategy will be to show that the projection of $P_0$ onto $T$ defines a subtree $T'\subseteq T$ with $|V(T')|\le \Delta^{O(h/\log h)}$.  Since $\rho$ is injective, $2^h=|V(P_0)|\le |V(T'\boxtimes P\boxtimes K_r)|$. Therefore,
  \[
    2^h = |V(P_0)| \le |V(T'\boxtimes P\boxtimes K_r)| \le r(2h+1)\Delta^{O(h/\log h)} = r\Delta\cdot 2^{O(\log h + h/\log h)}
  \]
  and rewriting this gives $\log(r\Delta) \ge h-O(\log h+h/\log h)=(1-o(1))\log n$, as required.  All that remains is to prove that $|V(T')|\le\Delta^{O(h/\log h)}$.  Since $T$ is max-degree-$\Delta$, so is $T'$, so it suffices to show that $\diam(T')=O(h/\log h)$, which we do next.

  For each $i\in\{0,\ldots,h-1\}$, let
  Let $d_i=\max\{\dist_T(v,w):v,w\in V(P_i)\}$.  Since $P_i$ is a path that contains exactly $2^{h-i}$ vertices, $d_i \le \diam(P_i) = 2^{h-i}-1$. Let $v$ and $w$ be two vertices of $P_i$ such that $\dist_T(v,w)=d_i$.  Observe that $P_{i+1}$ contains two vertices $x,y$ with $xv, yw\in E(G)$, which implies that $\dist_T(x,v)\le 1$ and $\dist_T(y,w)\le 1$.  By the triangle inequality, we have
  \[
    \dist_T(v,w) \le \dist_T(v,x)+\dist_T(x,y)+\dist_T(y,w) \le \dist_T(x,y)+2 \enspace .
  \]
  Therefore,
  \[
     d_{i+1} \ge \dist_T(x,y) \ge \dist_T(v,w)-2= d_i-2 \enspace ,
  \]
  for each $i\in\{0,\ldots,h-1\}$.  Beginning at $i=0$ and iterating this inequality implies that $d_i \ge d_0-2i$ for each $i\in\{0,\ldots,h\}$.
  Putting these upper and lower bounds together, we find that
  \[
    2^{h-i} \ge d_i \ge d_0-2i \enspace ,
  \]
  so
  \begin{equation}
    d_0 \le 2^{h-i}+2i  \label{d_i_bound}
  \end{equation}
  for each $i\in\{0,\ldots,h\}$.  Since $0\le d_0\le\diam(T)\le 2h$ it follows that $\lfloor d_0/4\rfloor\in\{0,\ldots,h\}$ so applying \cref{d_i_bound} with $i=\lfloor d_0/4\rfloor$ we obtain
  \[
    d_0 \le 2^{h-\lfloor d_0/4\rfloor} + d_0/2 \le 2^{h-d_0/4+1} + d_0/2 \enspace .
  \]
  Rewriting and simplifying gives $d_02^{d_0/4}\le 2^h$ which implies that $d_0 = O(h/\log h)$.  This completes the proof.
\end{proof}

\section{A Strengthening of \cref{treewidth_1_bounded_degree}}

By \cref{tree_partition_theorem}, \cref{treewidth_1_bounded_degree} is equivalent to \cref{easy_version}, which states that there are planar max-degree-$5$ graphs $G$ that are not subgraphs of $T\boxtimes P\boxtimes K_r$ unless the degree $\Delta$ of $T$ is large or the order $r$ of the clique $K_r$ is large.  In this section, we strengthen \cref{treewidth_1_bounded_degree} by showing that the result holds even if we allow the degree $\Delta$ of $T$ to be arbitrarily large.


\begin{lem}\label{hard_version}
  There exists a constant $\epsilon >0$ such that, for infinitely many positive integers $n$, there exists an $n$-vertex max-degree-$5$ planar graph $G$ such that, for any tree $T$, if $G\subseteq T\boxtimes P\boxtimes K_r$ then $r\in\Omega(\log^\varepsilon n)$.
\end{lem}

\todo[inline]{This whole section could use some work.  At the very least, we probably want a lemma/observation which states that if $W_{i-1}$ crosses an edge $vw$ of $T$ then $W_i$ visits at least one of $v$ or $w$.}

\begin{proof}
  Let $G$, $P_0,\ldots,P_h$, and $d_0,\ldots,d_h$, $\rho$, and $T'$ be exactly as in the proof of \cref{easy_version}.  The same arguments used there also apply here to establish that $|V(P)|\le 2h+1$ and that $\diam(T')\in O(h/\log h)$.  However, without any bound on the maximum degree of $T$, it is no longer possible to establish a useful upper bound on $|V(T')|$.

  Consider the lazy walk $W_0:=w_{0,1},\ldots,w_{0,2^h}$ in $T'$ obtained by projecting the path $P_0$ onto $T$.  Suppose that some node $x$ of $T'$ has degree $\delta$.  Then the walk $W_0$ must visit $x$ at least $\delta-1$ times.  To see why this is so, consider the components $T_1,\ldots,T_\delta$ of $T'-\{x\}$ and for each $i\in\{1,\ldots,\delta\}$ let $\tau(i)=\min\{j\in \{1,\ldots,2^h\}: w_j\in V(T_i)\}$.  Without loss of generality, we may assume that $\tau(1) < \tau(2) <\cdots < \tau(\delta)$.  In other words, $W_0$ visits some vertex of $T_i$ before visiting any vertex of $T_{i+1}$, for each $i\in\{1,\ldots,\delta-1\}$.  Since $w_{\tau(i)}$ and $w_{\tau(i+1)}$ are in different components of $T-x$, it is follows immediately that $x$ appears at least once in $w_{\tau(i)+1},\ldots,w_{\tau(i+1)-1}$ for each $i\in\{1,\ldots,\delta\}$.

  Since $W_0$ visits $x$ at least $\delta-1$ times, $\rho(P_0)$ contains at least $\delta-1$ vertices in $\{x\}\times V(P)\times V(K_r)$, which implies that $\delta \le (2h+1)r + 1$.  Therefore, $T'$ is a tree of height $O(\log h/\log\log h)$ and of maximum degree $O(hr)$.

  \begin{clm}\label{big_logeps_subdivision}
    There exists a fixed constant $\epsilon >0$
    $T'$ contains a subtree $T_0$ that is isomorphic to a subdivision of a complete $\lceil h^\epsilon\rceil$-ary tree of height $h'\ge\epsilon\log h/\log\log h$.
  \end{clm}

  \begin{proof}
    Do some Kraft-like counting.
  \end{proof}

  Let $\delta:=\lceil h^{\epsilon}\rceil+1$ and say that a node $x$ of $T_0$ is a \defin{branching node} if it has degree $\delta$ in $T_0$.

  \begin{clm}
    $T_0$ contains a path $X_0$ such that $W_0$ visits every branching node of $T_0-X_0$ at least $\delta$ times.
  \end{clm}

  \begin{proof}
    Say that a branching node $x$ of $T_0$ is \defin{light} if $x$ appears exactly $\delta-1$ times in $W_0$. Using the notation above, $x$ is light because $W_0$ visits $x$ exactly $\delta-1$ times and then remains in $T_\delta$.  Whenever this occurs at a node $x$, \defin{direct} the edge of $T_0$ that joint $x$ to $T_\delta$ away from $x$. Leave all other edges undirected.  In this way, each light branching node of $T_0$ has exactly one of its incident edges directed away from it.  Now observe that, if $x$ and $y$ are two light nodes of $T_0$ resulting in directed edges $e_x$ and $e_y$, then the path $P_{xy}$ in $T_0$ with endpoints $x$ and $y$ contains exactly one of $e_x$ or $e_y$.  We say that $x\prec y$ if $e_x$ is in $P_{xy}$ and $y\prec x$ if $e_y$ is in $P_x$.  Then the relationship $\prec$ is a total order on the light nodes of $T_0$ and there is a path $X_0$ that visits all these vertices in this order.
  \end{proof}


  For each $i\in\{0,\ldots,h'\}$, let $T_i$ be the subtree of $T$ visited by $W_i$ and let $Z_i$ be the set of branching nodes in $T_0$ that have exactly $h-i$ branching ancestors (including themselves).

  \begin{clm}
    For each $i\in\{0,\ldots,h\}$, $T_i$ contains every node in $\bigcup_{j=i}^{h'}Z_j$.
  \end{clm}

  \begin{proof}
    The proof is by induction on $i$. In the base case $i=0$ is true by the definition of $T_0$.  For $i\ge 1$, the inductive hypothesis implies that $W_{i-1}$ visits all branching nodes in $Z_{i-1}$. The walk $W_{i}$ ``follows'' $W_{i-1}$ in the sense that, for each $j\in\{1,\ldots,2^{h-i}\}$, $w_{i,j}$ is ajacent to, or equal to each of $w_{i-1,2j-1}$ and $w_{i-1,2j}$.  In particular, this implies that $W_i$ contains every vertex of $Z_i$. Since $W_i$ is a (lazy) walk in $T$, and $T$ is a tree, this implies that $W_i$ contains every branching ancestor of every node in $Z_i$.  Therefore, $W_i$ contains every vertex of $\bigcup_{j=i}^{h'} Z_j$, as required.
  \end{proof}

  At this point we are done.  Let $i:=\lfloor h'/2\rfloor$, so $i\in\Omega(h/\log h)$.  $T_0$ contains $\delta^i \in 2^{\Omega(h\log\delta/\log h)}$  nodes of $Z_i$.  For each $j\in\{0,\ldots,i\}$ at most two elements of $Z_i$ are visited fewer than $\delta$ by $W_j$.  Since $2i < |Z_i|$, there exists a node $x$ in $Z_i$ that is visited at least $\delta$ times by each of $W_0,\ldots,W_i$, for a total of $i\delta \in \Omega(h^{1+\epsilon}/\log h)$ times. This implies that $\rho$ maps at least $i\delta$ vertices of $G$ onto $\{x \}\times P\times K_r$, which implies that $i\delta\le (2h+1)r$, so $r\in\Omega(h^\epsilon/\log h)$.  The result now follows by taking $\varepsilon := \epsilon/2$.
\end{proof}


\section{An Open Problem}

We know that every planar graph $G$ is contained in a product of the form $H\boxtimes P\boxtimes K_3$ where $\tw(H)\le 3$. \cref{hard_version} states that, for any $r$, there exists a max-degree-$5$ planar that is not contained in any product of the form $T\boxtimes P\boxtimes K_r$ where $T$ is a tree.  This leaves the treewidth-$2$ case open:

\begin{quote}
  Is it the case that any max-degree-$\Delta$ planar graph $G$ is contained in a product of the form $H\boxtimes P\boxtimes K_r$ where $\tw(H)\le 2$ and $r$ is upper by some function of $\Delta$?
\end{quote}

\bibliographystyle{plainurlnat}
\bibliography{tp}

\end{document}
