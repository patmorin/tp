\documentclass{patmorin}
\listfiles
\usepackage{pat}
\usepackage{paralist}
\usepackage{dsfont}  % for \mathds{A}
\usepackage[utf8x]{inputenc}
\usepackage[T1]{fontenc}
\usepackage{skull}
\usepackage{paralist}
\usepackage{graphicx}
\usepackage{amssymb}
% \usepackage[noend]{algorithmic}
% \usepackage{bbm}  % needed for \mathbbm{1}
% \usepackage{listings}

\usepackage[normalem]{ulem}
\usepackage{cancel}
%\usepackage{enumitem}

\usepackage{todonotes}

% etoolbox allows for robust commands that don't need \protect, e.g.
% \newrobustcmd{\onesub}{\mathord{\includegraphics{figs/one-sub}}}
% \subsection{Approximate Voronoi Diagrams in $G^{\onesub}_k$}
\usepackage{etoolbox}

\usepackage[longnamesfirst,numbers,sort&compress]{natbib}

\usepackage[mathlines]{lineno}
\setlength{\linenumbersep}{2em}
% \linenumbers
% \rightlinenumbers
\newcommand*\patchAmsMathEnvironmentForLineno[1]{%
 \expandafter\let\csname old#1\expandafter\endcsname\csname #1\endcsname
 \expandafter\let\csname oldend#1\expandafter\endcsname\csname end#1\endcsname
 \renewenvironment{#1}%
    {\linenomath\csname old#1\endcsname}%
    {\csname oldend#1\endcsname\endlinenomath}}%
\newcommand*\patchBothAmsMathEnvironmentsForLineno[1]{%
 \patchAmsMathEnvironmentForLineno{#1}%
 \patchAmsMathEnvironmentForLineno{#1*}}%
\AtBeginDocument{%
\patchBothAmsMathEnvironmentsForLineno{equation}%
\patchBothAmsMathEnvironmentsForLineno{align}%
\patchBothAmsMathEnvironmentsForLineno{flalign}%
\patchBothAmsMathEnvironmentsForLineno{alignat}%
\patchBothAmsMathEnvironmentsForLineno{gather}%
\patchBothAmsMathEnvironmentsForLineno{multline}%
}

% Taken from
% https://tex.stackexchange.com/questions/42726/align-but-show-one-equation-number-at-the-end
\newcommand\numberthis{\addtocounter{equation}{1}\tag{\theequation}}


\setlength{\parskip}{1ex}

% Document-specific commands and math operators
\DeclareMathOperator{\depth}{depth}
\DeclareMathOperator{\tw}{tw}
\DeclareMathOperator{\dist}{dist}
\DeclareMathOperator{\diam}{diam}

% A version of \subseteq with a ~ on top
\newcommand\subsetcong{\mathrel{\text{%
    \setbox0\hbox{$\subseteq$}%
    \rlap{\hbox to \wd0{\hss\hss\hss\raisebox{1.5\height}{$\sim$}\hss}}\box0
}}}


% Some people prefer these (Pat doesn't)
\renewcommand{\le}{\leqslant}
\renewcommand{\ge}{\geqslant}

\title{\MakeUppercase{Bad News for Product Structure of Bounded-Degree Graphs}}

\author{%
  Vida Dujmović\thanks{Department of Computer Science and Electrical Engineering, University of Ottawa, Ottawa, Canada.}\, \thanks{Research partly supported by NSERC.} \quad
  Gwenaël Joret\thanks{D\'epartement d’Informatique, Universit\'e Libre de Bruxelles, Brussels, Belgium.} \quad
  Piotr Micek\thanks{Theoretical Computer Science Department, Faculty of Mathematics and Computer Science, Jagiellonian University, Krak\'ow, Poland.}\quad
  Pat Morin\thanks{School of Computer Science, Carleton University, Ottawa, Canada}\, \footnotemark[2] \quad
  David R. Wood\thanks{School of Mathematics, Monash University, Melbourne, Australia (\texttt{david.wood@monash.edu}). Research supported by the Australian Research Council.}
}

\date{}


\begin{document}
\maketitle


\begin{abstract}
  We describe an infinite family $\mathcal{G}$ of planar graphs of maximum degree $5$ such that, if an $n$-vertex member of $\mathcal{G}$ is a subgraph of the strong product of a graph $H$ of treewidth $t$ and maximum degree $\Delta$, a path $P$ and a clique $K$ of order $c$, then $t\Delta c \ge 2^{\Omega(\sqrt{\log\log n})}$.
\end{abstract}

\section{Introduction}

Recently, product structure theorems have been a key tool in resolving a number of longstanding open problems on planar graphs.  Roughly, a \defin{product structure theorem} for a graph family $\mathcal{G}$ states that every graph in $\mathcal{G}$ is isomorphic to a subgraph of the product of two or more ``simple'' graphs.  As an example, there are a number of graph classes $\mathcal{G}$ for which there exists integers $t$ and $c$ such that, for each $G\in\mathcal{G}$ there is a graph $H$ of treewidth $t$ and a path $P$ such that $G$ is isomorphic to a subgraph of the strong product of $H$, $P$, and a clique $K$ of order $c$.\footnote{A \defin{tree decomposition} of a graph $H$ is a collection $\mathcal{T}:=(B_x:x\in V(T))$ of subsets of $V(H)$ indexed by the nodes of some tree $T$ such that
\begin{inparaenum}[(i)]
  \item for each $v\in V(H)$, the induced subgraph $T[x\in V(T):v\in B_x]$ is connected; and
  \item for each edge $vw\in E(H)$, there exists some $x\in V(T)$ with $\{v,w\}\subsetcong B_x$.
\end{inparaenum}
The \defin{width} of such a tree decomposition is $\max\{|B_x|:x\in V(T)\}-1$. The \defin{treewidth} of $H$ is the minimum width of any tree decomposition of $H$.}\footnote{The \defin{strong product} $G_1\boxtimes G_2$ of two graphs $G_1$ and $G_2$ is the graph with vertex set $V(G_1\boxtimes G_2):=V(G_1)\times V(G_2)$ and that includes the edge with endpoints $(v,x)$ and $(w,y)$ if and only if
\begin{inparaenum}[(i)]
  \item $vw\in E(G_1)$ and $x=y$;
  \item $v=w$ and $xy\in E(G_2)$; or
  \item $vw\in E(G_1)$ and $xy\in E(G_2)$.
\end{inparaenum}
}
This is typically written as $G\subsetcong H\boxtimes P\boxtimes K_c$, where the notation $G_1\subsetcong G_2$ is used to mean that $G_1$ is isomorphic to some subgraph of $G_2$.  See references \cite{dujmovic.joret.ea:planar,dujmovic.morin.ea:structure,krauthgamer.lee:intrinsic,ueckerdt.wood.ea:improved,bose.morin.ea:optimal,campbell.clinch.ea:product,illingworth.scott.ea:alon} for examples.

In some applications of product structure theorems, it is helpful if, in addition to having treewidth $t$, the graph $H$ has additional properties, possibly inherited from $G$.  For example, one very useful version of the planar graph product structure theorem states that for every planar graph $G$ there exists a \emph{planar} graph $H$ of treewidth $3$ and a path $P$ such that $G\subsetcong H\boxtimes P\boxtimes K_3$.  The planarity of $H$ in this result has been leveraged to obtain better constants and even asymptotic improvements for graph colouring and layout problems, including queue number \cite{dujmovic.joret.ea:planar}, $p$-centered colouring \cite{debski.felsner.ea:improved}, and $\ell$-vertex ranking \cite{bose.dujmovic.ea:asymptotically}.

In this vein, the authors have been repeatedly asked the following question, which asks if $H$ can have bounded degree whenever $G$ does:
\begin{quote}
  Let $\mathcal{G}_\Delta$ be the family of planar graphs of maximum degree $\Delta$.  Do there exist functions $t:\N\to\N$, $d:\N\to\N$, and $c:\N\to\N$ such that, for each $G\in\mathcal{G}_\Delta$ there exists a graph $H$ of treewidth $t(\Delta)$ and maximum degree $d(\Delta)$ such that $G\subsetcong H\boxtimes P\boxtimes K_{c(\Delta)}$?
\end{quote}
In this note we show that the answer to this question is no, even when $\Delta=5$.

\begin{thm}\label{main_thm}
  There exists $\epsilon > 0$ such that, for infinitely many integers $n\ge 1$, there exists an $n$-vertex planar graph $G$ of maximum degree $5$ such that, for any graph $H$ of treewidth $t$ and maximum degree $\Delta$, any path $P$, and any integer $c$, $G\subsetcong H\boxtimes P\boxtimes K_c$ implies $t\Delta c \ge 2^{\Omega(\sqrt{\log\log n})}$.
\end{thm}

The graph family $\mathcal{G}:=\{G_h:h\in\N\}$ that establishes \cref{main_thm} consists of complete binary trees of height $h$ augmented with edges to form, for each $i\in\{1,\ldots,h\}$, a path that contains all vertices of depth $i$.  See \cref{G_5}.

\begin{figure}
  \begin{center}
    \includegraphics{figs/g_5}
  \end{center}
  \caption{The graph $G_5$ from the graph family $\{G_h:h\in\N\}$ that establishes \cref{main_thm}.}
  \label{G_5}
\end{figure}

\section{Proof of \cref{main_thm}}

Throughout this paper, all graphs $G$ are simple and undirected with vertex set $V(G)$ and edge set $E(G)$.  For a set $S$, $G[S]$ denotes the subgraph of $G$ induced by $S\cap V(G)$ and $G-S:=G[V(G)\setminus S]$.  For any $v\in V(G)$, let $N_G(v):=\{w:vw\in E(G)\}$ and for any $S\subseteq V(G)$, let $N_G(S):=\bigcup_{v\in S} N_G(v)\setminus S$.  We write $G_1\cong G_2$ if $G_1$ and $G_2$ are isomorphic and $G_1\subsetcong G_2$ if $G_1$ is isomorphic to some subgraph of $G_2$.

\subsection{Partitions}

Let $G$ and $H$ be graphs.  An \defin{$H$-partition} $\mathcal{H}:=\{B_x:x\in V(H)\}$ of $G$ is a partition of $V(G)$ whose parts are indexed by the vertices of $H$ with the property that, if $vw$ is an edge of $G$ with $v\in B_x$ and $w\in B_y$ then $x=y$ or $xy\in E(H)$.  The \defin{width} of $\mathcal{H}$ is the size of its largest part; that is, $\max\{|B_x|:x\in V(H)\}$.

If $H$ is in a class $\mathcal{G}$ of graphs then we may call $\mathcal{H}$ a $\mathcal{G}$-partition of $G$.  Specifically, if $H$ is a tree, then $\mathcal{H}$ is a \defin{tree-partition} of $G$ and if $\mathcal{H}$ is a path, then $\mathcal{H}$ is a \defin{path-partition} of $G$.  A path-partition $\mathcal{P}:=\{P_x:x\in V(P)\}$ of $G$ is also referred to as a \defin{layering} of $G$ and the parts of $\mathcal{P}$ are referred to as \defin{layers}.  A set of layers $\{P_{x_1},\ldots,P_{x_q}\}\subseteq\mathcal{P}$ is \defin{consecutive} if $P[\{x_1,\ldots,x_q\}]$ is connected.

As in previous works, we make use of the following relationship between $H$-partitions and strong products, which follows immediately from the preceding definitions.

\begin{obs}\label{partitions_vs_products}
  For any integer $c\ge 1$, and any graphs $G$, $H$, and $J$,  $G\subsetcong H\boxtimes J\boxtimes K_c$ if and only if $G$ has an $H$-partition $\mathcal{H}:=\{B_x:x\in V(H)\}$ and a $J$-partition $\mathcal{J}:=\{C_y:y\in V(J)\}$ such that $|B_x\cap C_y|\le c$, for each $(x,y)\in V(H)\times V(J)$.
\end{obs}

The following important result of \citet{ding.oporowski:some} (see also \citet{wood:on}) allows us to focus on the case where the first factor in our product is a tree.

\begin{thm}[\citet{ding.oporowski:some}]\label{dingy}
  If $H$ is a graph with maximum degree $\Delta$ and treewidth $t$,\todo{DW suggests writing "treewidth less than $t$" here, but that's not accurate. If he means "treewidth at most $t$", that seems redundant} then $H$ has a tree-partition of width at most $24\Delta t$.
\end{thm}

\begin{cor}\label{ding_translation}
  If $G\subsetcong H\boxtimes P\boxtimes K_c$ where $H$ has treewidth $t$ and maximum degree $\Delta$ then there exists a tree $T$ such that $G\subsetcong T\boxtimes P\boxtimes K_{24c\Delta t}$.
\end{cor}

\begin{proof}
  By \cref{dingy}, $H$ has a tree-partition $\mathcal{T}:=\{B_x:x\in V(T)\}$ of width at most $24\Delta t$ so, by \cref{partitions_vs_products}, $H \subsetcong T\boxtimes K_{24\Delta t}$.  Therefore, $G\subsetcong T\boxtimes K_{24\Delta t}\boxtimes P\boxtimes K_c \cong T\boxtimes P\boxtimes K_{24c\Delta t}$.
\end{proof}

The \defin{length} of a path is the number of edges in it. For any two vertices $v,w\in V(G)$, $\dist_G(v,w)$ denotes the minimum length of a path in $G$ that contains $v$ and $w$, or $\dist_G(v,w):=\infty$ if $v$ and $w$ are in different connected components of $G$. For any $R\subsetcong V(G)$, the \defin{diameter} of $R$ is $\diam_G(R):=\max\{\dist_G(v,w):v,w\in R\}$.

\begin{obs}\label{diameter_spread}
  Let $G$ be any graph, let $R\subseteq V(G)$, and let $\mathcal{L}$ be any layering of $G$.  Then there exist a layer $L\in\mathcal{L}$ such that $|R\cap L|\ge |R|/(\diam_G(R)+1)$.
\end{obs}

\begin{proof}
  By the definition of layering, the vertices in $R$ are contained in a set of at most $\diam_G(R)+1$ consecutive layers of $\mathcal{L}$. The result then follows from the Pigeonhole Principle.
\end{proof}


We also make use of the following basic fact about tree-partitions:

\begin{obs}\label{tree_thingy}
  Let $G$ be a graph, let $\mathcal{T}:=(B_x:x\in V(T))$ be a tree-partition of $G$, let $x\in V(T)$, and let $v,w\in N_G(B_x)$ be in the same component of $G-B_x$.  Then $T$ contains an edge $xy$ with $v,w\in B_y$.
\end{obs}

\begin{proof}
  Suppose that $v\in B_y$ and $w\in B_z$ for some $y,z\in V(T)$.  Since $v,w\in N_G(B_x)$, $T$ contains the edges $xy$ and $xz$.  All that remains to is to show that $y=z$ so, for the purpose of contradiction, assume $y\neq z$.  Since $v$ and $w$ are in the same component of $G-B_x$, $G$ contains a path from $v$ to $w$ that avoids all vertices in $B_x$, which implies that $T$ contains a path $P_{yz}$ from $y$ to $z$ that does not include $x$.  This is a contradiction since then $P_{yz}$ and the edges $xy$ and $yz$ form a cycle in $T$, but $T$ is a tree.
\end{proof}

\subsection{Percolation in Binary Trees}


The \defin{depth} of a vertex $v$ in a rooted tree $T$ is the length of the path $P_T(v)$ from $v$ to the root of $T$.  Each vertex $a\in V(P_T(w))$ is an \defin{ancestor} of $v$, and $v$ is a \defin{descendant} of each vertex in $V(P_T(w))$.  We say that a set $B\subset V(T)$ is \defin{unrelated} if the no vertex of $B$ is an ancestor of any other vertex in $B$.  For each $h\in\N$, let $T_h$ denote the complete binary tree of height $h$; i.e., the rooted ordered tree with $2^h$ leaves, each having depth $h$ and in which each non-leaf vertex has exactly two children.  Note that the ordering of $T_h$ induces an ordering on any unrelated set $B\subseteq V(T_h)$, which we refer to as the \defin{left-to-right ordering}.

We use the following two percolation-type results for $T_h$.

\begin{lem}\label{one_path}
  Let $h\ge 1$
  % let $T_h$ be a complete binary tree of height $h$ with root $r$,
  and let $S\subseteq V(T_h)$ with $1\le |S|< 2^h$. Then there exists a vertex $v$ of $T_h$ such that
  \begin{compactenum}[(i)]
    \item the depth of $v$ is at most $\log_2|S|+1$;
    \item the parent of $v$ is in $S\cup\{r\}$; and
    \item $T_h-S$ contains a path from $v$ to a leaf of $T_h$.
  \end{compactenum}
\end{lem}

\begin{proof}
  The proof is by induction on $h$.  When $h=1$, $|S|\le 1$. In particular, at least one child $v$ of $r$ is not in $S$.  The depth of $v$ is $1\le \log_2|S|+1$, so $v$ satisfies (i).  The parent of $v$ is $r\in S\cup\{r\}$, so $v$ satisifies (ii).  $T_1-S$ contains a length-$0$ path from $v$ to itself (a leaf of $T_1$), so $v$ satisfies (iii).

  For $h\ge 2$, let $\ell$ be the maximum integer such that $S\cup\{r\}$ contains all $2^\ell$ vertices of depth $\ell$.  Observe that $2^\ell \le |S|$, so $\ell \le \log_2 |S| < h$.  Let $L$ be the set of $2^{\ell+1}$ depth-$(\ell+1)$ vertices in $T_h$.  By the Pigeonhole Principle some vertex $r'\in L$ is the root of a complete binary tree $T'$ with root $r'$ of height $h-\ell$ with $|S\cap V(T')| \le |S|/2^{\ell+1} < 2^{h-\ell+1}$.  Applying induction on $T'$ and $S':=S\cap V(T')$ we obtain a vertex $v'$ of depth at most $\ell+1+\log_2(|S'|)+1 \le \log_2|S|+1$ whose parent is in $S\cup\{r'\}$, and such that $T_h-S$ contains a path from $v'$ to a leaf of $T_h$.  Thus $v'$ satisfies requirements (i) and (iii).  If the parent of $v'$ is in $S$ then $v'$ also satisfies requirement (ii) and the lemma is proven, with $v:=v'$.  Otherwise, the parent of $v'$ is $r'$, in which case $r'$ satisfies requirements (i)--(iii) and we are done, with $v:=r'$.
\end{proof}

\begin{lem}\label{two_paths}
  Let $h\ge 2$
  % let $T_h$ be a complete binary tree of height $h$ with root $r$,
  and let $S\subseteq V(T_h)$ with $1\le |S|< 2^{h-1}$. Then there exists two unrelated vertices $v_1$ and $v_2$ of $T_h$ such that, for each $i\in\{1,2\}$
  \begin{compactenum}[(i)]
    \item the depth of $v_i$ is at most $\log_2|S|+2$;
    \item the parent of $v_i$ is in $S\cup\{r\}$; and
    \item $T_h-S$ contains a path from $v_i$ to a leaf of $T_h$.
  \end{compactenum}
\end{lem}

\begin{proof}
  Let $T_1$ and $T_2$ be the two maximal subtrees of $T_h$ rooted at the children $r_1$ and $r_2$, respectively of $r$. (Each of $T_1$ and $T_2$ is a complete binary tree of height $h-1$.)  For each $i\in\{1,2\}$, let $S_i:=S\cap V(T_i)$.  Since $|S_i|\le |S|< 2^{h-1}$, we can apply \cref{one_path} to $T_i$ and $S_i$ to obtain a vertex $v_i'\in V(T_i)$ of depth at most $1+\log_2|S_i|+1 \le \log_2 |S| + 2$ and such that $T_h-S$ contains a path from $v_i'$ to a leaf of $T_h$.  Therefore, $v_i'$ satisfies (i) and (iii).  Furthermore, the parent of $v_i'$ is in $S\cup\{r_i\}$.  If the parent of $v_i'$ is in $S$, then $v_i'$ also satisfies (ii), so we set $v_i:=v_i'$.  If the parent of $v_i'$ is not in $S$, then the parent of $v_i'$ is $r_i\not\in S$ and $r_i$ satisfies (i)--(iii), so we set $v_i:=r_i$.  Finally, since $v_1\in V(T_1)$ and $v_2\in V(T_2)$, $v_1$ and $v_2$ are unrelated.
\end{proof}

\subsection{A Connectivity Lemma}

The \defin{$x\times y$ grid} $G_{x\times y}$ is the graph with vertex set $V(G_{x\times y}):=\{1,\ldots,x\}\times\{1,\ldots,y\}$ and that contains an edge with endpoints $(x_1,y_1)$ and $(x_2,y_2)$ if and only if $|x_1-x_1|+|y_1-y_2|=1$.  An edge of $G_{x\times y}$ is \defin{horizontal} if its two endpoints agree in the second (y) coordinate.  For each $i\in\{1,\ldots,x\}$, the vertex set $\{i\}\times\{1,\ldots,y\}$ is called \defin{column $i$} of $G_{x\times y}$.  A set $C$ of columns is \defin{consecutive} if $G_{x\times y}[\cup C]$ is connected.


\begin{lem}\label{grid_connectivity}
  Let $x,y,p\ge 1$ be integers, let $G$ be a graph obtained by subdividing horizontal edges of $G_{x\times y}$, and let $S\subsetcong V(G)\setminus V(G_{x\times y})$ be a set of subdivision vertices of size $|S|< py$.  Then some component of $G-S$ contains at least $x/p$ consecutive columns of $G_{x\times y}$.
\end{lem}

\begin{proof}
  For each $i\in\{1,\ldots,x-1\}$, in order to separate column $i$ from column $i+1$, $S$ must contain at least $y$ subdivision vertices on the horizontal edges between columns $i$ and $i+1$.  Since $|S|< py$, this implies that there are at most $p-1$ values of $i\in\{1,\ldots,x-1\}$ for which columns $i$ and $i+1$ are in different components of $G-S$. These at most $p-1$ values of $i$ partition $\{1,\ldots,x\}$ into at most $p$ intervals, one of which contains at least $x/p$ consecutive columns that are contained in a single component of $G-S$.
\end{proof}

\subsection{The Proof}


% We will construct a family $\mathcal{G}:=\{G_{h}:h\in\N\}$ of planar graphs with the following property:


Recall that, for each $h\in\N$, $G_{h}$ is the planar supergraph of the complete binary tree $T_h$ of height $h$ obtained by adding the edges of a path $P_i$ that contains all vertices of depth $i$, in left-to-right order, for each $i\in\{1,\ldots,h\}$.   Since $T_h$ is a spanning subgraph of $G_h$, the \defin{depth} of a vertex $v$ in $G_h$ refers to the depth of $v$ in $T_h$.  The \defin{height} of a depth-$d$ vertex of $T_h$ is $h-d$.\todo{DW asks if we need both depth and height. No, but without it we get $h-d$ appearing more often than I'd like.}  We are now ready to show that for any tree $T$, any path $P$, $G_{h}\subsetcong T\boxtimes P\boxtimes K_c$ implies that $c\ge 2^{\Omega(\log h)}$.  By \cref{ding_translation}, this is enough to prove \cref{main_thm}.

% \todo[inline]{This is probably the place to write a proof outline.}

We now outline our proof strategy.  Recall \cref{partitions_vs_products} which states that if $G_h\subseteq T\boxtimes P\boxtimes K_c$ then $G_h$ has a tree-partition $\mathcal{T}:=\{B_x:x\in V(T)\}$ and a path-partition (i.e., layering) $\mathcal{P}:=\{P_y:y\in V(P)\}$ with $|B_x\cap P_y|\le c$ for each $(x,y)\in V(T)\times V(P)$. Our strategy is to first show that $\mathcal{T}$ has some part $B_{x_1}$ that contains an unrelated set $\mathcal{R}_1$ of small depth and size at least $\alpha h$ for some constant $\alpha >0$.  Next we show that $x_1$ has a neighbour $x_2$ that contains an unrelated set $\mathcal{R}_2$ that is smaller than $\mathcal{R}_1$ by a factor of at most $b/2$ (for some small $b>2$), but whose elements can be paired off so that each pair has a common ancestor in $R_1$ that is of distance at most $a$ from both vertices in the pair (for some small $a>0$).  By iterating this argument $t$ times, we obtain a sequence of unrelated sets $\mathcal{R}_1,\ldots,\mathcal{R}_{t+1}$ where the elements in each $\mathcal{R}_i$ are partitioned into groups of size $2^{i-1}$ so that each group $R$ has a common ancestor in $\mathcal{R}_1$ that is of distance at most $a(i-1)$ from all elements in $R$.  Therefore $\mathcal{R}_{t+1}\subseteq B_{x_{t+1}}$ contains an unrelated set of size $2^{t}$ whose diameter is at most $2at$.  \cref{diameter_spread} then implies that $|B_{x_{t+1}}\cap P_y|\ge 2^{t}/2at$ for some $y\in V(P)$.  By choosing  $t\approx\log_b |R_1|\ge \log_{b}(\alpha h)$, this implies that $c\ge |B_{x_{t+1}}\cap P_y|\gtrapprox 2^{\log_{b}(\alpha h)}/2a(\log_{b}(\alpha h))=h^{\alpha/\log b-o(1)}$.

We now proceed with the details of the strategy outlined above.  The next two lemmas will be used to obtain the set $\mathcal{R}_1$ that allows us to start the argument.

\begin{lem}\label{small_depth_separator}
  There exists a function $i_0:\R_{>0}\to\N\in O(\log n)$ such that for all $h\in\N$: If $S\subseteq V(G_h)$ with $|S|\le ch$ and such that $G_h-S$ have no component with more than $|V(G_h)|/2$ vertices then, for each $i\in\{i_0(ch+1),\ldots,h\}$, $S$ contains at least one vertex of depth $i$.
\end{lem}

\begin{proof}
  Let $C$ be the vertex set of a component of $G_h-S$ that contains the largest number of leaves of $T_h$.  For each $i\in\{0,\ldots,h\}$, let $C_i:=C\cap V(P_i)$. We will show that, for each $i\in\{i_0(ch+1),\ldots,h\}$, $C_i$ is non-empty but does not contain all $2^i$ vertices of depth $i$.  Therefore $S\cap V(P_i)\supset N_G(C_i)\cap V(P_i)\neq\emptyset$, i.e., $S$ contains at least one vertex of depth $i$.

  For each $i\in\{0,\ldots,h-1\}$, the vertices in $C_{i+1}$ are adjacent to at least $|C_{i+1}|/2$ vertices of $P_{i}$, so $ch \ge |S| \ge |N_G(C_{i+1})\cap V(P_i)\setminus C_{i}| \ge |C_{i+1}|/2 - |C_{i}|$ and therefore, $|C_{i}| \ge |C_{i+1}|/2 - ch$.  Similarly, the vertices in $C_i$ are adjacent to at least $2|C_i|$ vertices in $P_{i+1}$, so $|C_i|\le (|C_{i+1}| + ch)/2$.  Iterating these these two inequalities with $i=h-1,h-2,\ldots,0$ gives the upper and lower bounds
  \begin{equation}
    |C_h|/2^{h-i} - 2ch \le C_{i} \le |C_h|/2^{h-i} + ch   \enspace ,
    \label{C_i_bounds}
  \end{equation}
  for each $i\in\{0,\ldots,h-1\}$.

  By definition, $2^h >|V(G)|/2 \ge |C| := \sum_{i=0}^h |C_h| \ge 2|C_h|-ch^2-1$, therefore $|C_h| < 2^{h-1} + ch^2/2 + 1/2$.  Next, the at most $ch$ vertices of $S$ partition $V(P_h)\setminus S$ into at most $ch+1$ connected components, so $|C_h| \ge (2^h-ch)/(ch+1) \ge 2^{h}/(ch+1) - 1$.  Therefore, by \cref{C_i_bounds}
  \[
      2^{i-\log(ch+1)} - O(ch^2) \le \frac{2^h/(ch+1)-1}{2^{h-i}} - 2ch \le |C_i|
      \le \frac{2^{h-1}+ch^2/2 + 1/2}{2^{h-i}} + ch \le 2^{i-1} + O(ch)
      \enspace .
  \]
  In particular, $0 < |C_i| < 2^i$ for $i\ge i_0(ch+1)= \log (ch+1) + \log(2ch)+O(1)$.
\end{proof}


The following lemma shows that any tree-partition of $G_h$ must have a part with a large unrelated set that is far from the leaves of $T_h$.

\begin{lem}\label{startup}
  For any $\alpha\in(0,1/4)$, there exists $h_0$ such that the following is true, for all integers $h\ge  h_0$ and all $c\in[1,h]$.  If $\mathcal{T}:=\{B_x:x\in V(T)\}$ is a tree-partition of $G_h$ of width less than $ch$ then there exists a node $x\in V(T)$ and a subset $R\subseteq B_x$ such that
  \begin{compactenum}[(i)]
    \item $R$ is unrelated;
    \item $|R|\ge \alpha h/c$; and
    \item Each vertex in $R$ has height at least $\alpha h$
  \end{compactenum}
\end{lem}


\begin{proof}
  It is well-known and easy to show that there exists a node $x$ of $T$ such that $G-B_x$ has no component with more than $|V(G_h)|/2$ vertices. Let $Y$ be the set of vertices in $B_x$ that have height at least $h/4$.  By \cref{small_depth_separator}, $|Y|\ge 3h/4 - O(\log (ch+1))$.

  Let $T_Y$ be the minimal (connected) subtree of $T_h$ that spans $Y$, and let $L$ the set of leaves of $T_Y$.  Observe that $L\subseteq Y$ is an unrelated set. Therefore, $L$ satisfies (i) and, by definition, each vertex in $L$ has height at least $h/4 > \alpha h$, so $L$ satisfies (iii).  If $|L|\ge \alpha h \ge \alpha h/c$ then $L$ also satisfies (ii).  In this case, we can take $R:=L$ and we are done.  We now assume that $|L|< \alpha h$.

  Let $Z$ consist of all vertices in $V(T_h)\setminus V(T_Y)$ whose parents are in $Y\setminus L$.   Observe that $Z$ is a unrelated set of vertices each having height at least $h/4$. For each $v$ of $T_Y$, let $d_v$ denote the number of children of $v$ in $T_Y$.  Then,
  \[
     \sum_{v\in Y\setminus L} (d_v-1)
     \le \sum_{v\in V(T_Y)\setminus L} (d_v-1)
     = |L|-1 \enspace ,
  \]
  where the second equality is a standard fact about rooted trees.
  Rewriting this, we get $\sum_{v\in Y\setminus L} {d_v} < |Y\setminus L| + |L| = |Y|$.  On the other hand, each $v\in Y\setminus L$ contributes $2-d_v$ vertices to $Z$, so
  \[
    |Z| = \sum_{v\in Y\setminus L} (2-d_v) \enspace .
  \]
  Combining these two formulas, we obtain
  \[
    |Z| \ge 2|Y\setminus L| - |Y| = |Y| - 2|L|
    \ge 3h/4-O(\log(ch+1)) - 2\alpha h
    % = (3/4-2\alpha)h - O(\log(ch+1))
    \ge h/4-O(\log(ch+1)) \enspace .
  \]
  Refer to \cref{paths}.  For each $r\in Z$, \cref{one_path} implies that $r$ has a descendant $v$ whose parent is in $B_x$, that has height at least at least $h/4-O(\log(ch+1))$ and such that $T_h-B_x$ contains a path $P_{v}$ from $v$ to a leaf of $T_h$.  Form the set $Z'$ by replacing each vertex $r\in Z$ with the vertex $v$ just described.  Again, $Z'$ is a unrelated set.  Since $\alpha < 1/4$, for sufficiently large $h$, $|Z'|\ge h/4 - O(\log(ch+1)) \ge \alpha h$ and each vertex in $Z$ has height at least $h/4 - O(\log(ch+1)) \ge \alpha h$.

  \begin{figure}
    \begin{center}
      \includegraphics{figs/paths}
    \end{center}
    \caption{A step in the proof of \cref{startup}.}
    \label{paths}
  \end{figure}

  Now observe that the union of the paths in  $\{P_{v}:v\in Z'\}$ and the paths $P_{h-\lceil\alpha h\rceil+1},\ldots,P_{h}$ contains a subgraph $G'$ isomorphic to a graph that can be obtained from the grid $G_{\lceil \alpha h\rceil\times\lceil \alpha h\rceil}$ by subdividing horizontal edges.  Since $S$ does not contain any vertex of $P_{v}$ for any $v\in Z'$,  $S\cap V(G')$ contains only vertices corresponding to subdivision vertices.  Therefore, by \cref{grid_connectivity}, some component of $G-B_x$ contains a subset $R\subseteq Z'$ of size at least $\alpha h/c$.  Each element in $R$ has a parent in $B_x$ so, by \cref{tree_thingy} some neighbour $y$ of $x$ in $T$ has a bag $B_y$ that contains all of $R$.  This completes the proof.
\end{proof}


A set $\mathcal{R}:=\{R_1,\ldots,R_q\}$ of subsets of $V(T_h)$ is \defin{$(k,\ell,m)$-compact}\todo{pick a better name. DW suggests "rich"} if it has the following properties:

\begin{compactenum}
  \item For each $i\in\{1,\ldots,q\}$, $|R_i|\ge k$.
  \item For each $i\in\{1,\ldots,q\}$ there exists a common ancestor $a_i$ of $R_i$ such that $\dist_T(v,a_i)\le\ell$ for each $v\in R_i$.
  \item $a_1,\ldots,a_q$ are unrelated and each has height at least $m$.
\end{compactenum}

We say that a vertex $v$ of $T_h$ is \defin{compatible} with $S\subseteq V(T_h)$ if the parent of $v$ is in $S$ and $T_h-S$ contains a path from $v$ to a leaf of $T_h$.


\begin{lem}\label{compatible_set}
  Let $\mathcal{R}:=\{R_1,\ldots,R_q\}$ be a $(k,\ell,m)$-compact set,  and let $S\supseteq \cup\mathcal{R}$ have size $1\le |S|< 2^{m-\ell-2}$.  Then, there exists a $(2k,\ell + \log_2|S|+2,m)$-compact set $\mathcal{R}':=\{R_1',\ldots,R_q'\}$ that is compatible with $S$.
\end{lem}

\begin{proof}
  For each $i\in\{1,\ldots,r\}$ and each $r\in R_i$, replace $r$ with the descendants $v_1$ and $v_2$ of $r$ described in \cref{two_paths} and call the resulting set $R_i'$.   Then $|R_i'|=2|R_i|\ge 2k$ and $\dist_{T_h}(v,a_i)\le \ell+\log_2|S|+2$ for each $v\in R_i'$.  Therefore $\mathcal{R}':=\{R_1',\ldots,R_q'\}$ is a $(2k,\ell + \log_2(|R|)+2,m)$-compact set.
\end{proof}

The next lemma is the last ingredient in the proof of \cref{main_thm}.
\begin{lem}\label{big_lemma}
  Let $\mathcal{T}:=\{B_x:x\in V(T)\}$ be a tree-partition of $G_h$ and let $\mathcal{P}:=\{P_y:y\in V(P)\}$ be a path-partition of $G_h$.  Then there exists $(x,y)\in V(T)\times V(P)$ such that $|B_x\cap P_y| \ge 2^{\Omega(\sqrt{\log h})}$.
\end{lem}

\begin{proof}
  Let $x\in V(T)$ be a node that maximizes $|B_x|$.  Then $\diam_{G_h}(B_x)\le\diam_{T_h}(B_x) \le 2h+1$ so, by \cref{diameter_spread}, $|B_x\cap P_y|\ge |B_x|/(2h+1)$.  If $|B_x|\ge h2^{\sqrt{\log h}}$ then there is nothing more to prove, so we may assume that $|B_x| < ch$ for some $c\le 2^{\sqrt{\log h}}$.

  By \cref{startup}, $T$ contains a node $x_1$ such that $B_{x_1}$ contains a unrelated set $R$ of size $q_1:=|R|\ge \alpha h/c$ where each vertex in $R$ has height at least $m:=m_1:=\alpha h$.  Let  $\mathcal{R}_1':=\{\{v\}:v\in R\}$.  By definition $\mathcal{R}_1'$ is a $(1,0,m)$-compact set.

  By \cref{compatible_set} (applied to $\mathcal{R}_1'$ and $B_{x_1}$), $T_h$ contains a $(2,\log_2(ch)+2,m)$-compact set $\mathcal{R}_1$ of size $q_1$ that is compatible with $B_{x_1}$.  Let $m_2:=m_1-\lfloor\log_2(ch)\rfloor -2$.  For each $v\in\cup\mathcal{R}_1$, $T_h-B_{x_1}$ contains a path $P_v$ from $v$ to a leaf of $T_h$.  The union of the paths in $P_{h-m_2+1},\ldots,P_{h}$ and the paths in $\mathcal{C}_1:=\{P_v:v\in\cup\mathcal{R}_1\}$ contains a subgraph isomorphic to a graph that can be obtained from $G_{2q_1\times m_2}$ by subdividing horizontal edges.  Therefore, by \cref{grid_connectivity}, some component of $G-S_1$ contains at least $2q_1m_2/ch$ of the paths in $\mathcal{C}_1$.  Furthermore, these paths are consecutive in the left-to-right ordering of $\mathcal{C}_1$.  This implies that there is a subset $\mathcal{R}_2'\subseteq \mathcal{R}_1$ of size $q_2\ge \lfloor(2q_1m_2/ch-1)/2\rfloor$ that is contained in a single component of $G-B_{x_1}$. By \cref{tree_thingy}, there is a neighbour $x_2$ of $x_1$ in $T$ with $\cup\mathcal{R}_2'\subsetcong B_{x_2}$.

  The bound on the size $q_2$ of $\mathcal{R}_2'$ is awkward, so we simplify it as follows:
  \[
    q_2 \ge \left\lfloor\left(\frac{2q_1m_2}{ch} - 1\right) /2 \right\rfloor
        \ge \frac{q_1m_2}{ch} - 1 \ge \frac{\alpha q_2}{2c} - 1
  \]
  provided that $m_2 \ge\alpha h/2$.  This latter condition is true for any fixed $\alpha$ and all sufficiently large $h$, since $m_2 = m_1 - O(\log h) \ge \alpha h - O(\log h)$.

  We can now iterate this process of obtaining $\mathcal{R}_{i+1}'\subseteq B_{x_{i+1}}$ from $\mathcal{R}_i'\subseteq B_{x_i}$. Let $t:= \lfloor \min\{\log_{2c/\alpha} q_1-2,\alpha h/(2\log_2(ch)+4)-2\}\rfloor$.\footnote{The second term in the definition of $t$ ensures that $q_{i+1}\ge \alpha q_i/2c-1$ for each $i\in\{1,\ldots,t\}$.}   Then after $t$ such iterations we obtain a $(2^t,t(\log_2(ch)+2),r_1)$-compact set $\mathcal{R}_{t+1}'\subseteq B_{x_t}$ of size at least $q_{t+1} \ge q_1(\alpha/2c)^t-2 \ge 1$.\footnote{$q_{t+1} \ge q_t(\alpha/2c) - 1 \ge q_1(\alpha/2c)^t - \sum_{i=0}^t (\alpha/2c)^i \ge q_1(\alpha/2c)^t - 2$ for $c\ge 1$}

  Let $R$ be one of the sets in $\mathcal{R}_{t+1}'$.  Since $\mathcal{R}_{t+1}'$ is $(2^t,t(\log_2(ch)+2),m)$-compact, $|R|\ge 2^t$ and all vertices in $R$ have a common ancestor $a$ whose distance to each element of $R$ is at most $t(\log_2(ch)+2)$.  Therefore, $\diam_{G_h}(R)\le\diam_{T_h}(R)\le 2t(\log_2(ch)+2)$. By \cref{diameter_spread}, there exists some $P_y\in\mathcal{P}$ with
  \begin{align*}
    |B_{x_t}\cap P_y|
    & \ge |R\cap P_y| \\
    & \ge \frac{|R|}{\diam_{T_h}(R)} \\
    & \ge \frac{2^t}{2t(\log_2(ch)+2)+1} \\
    & = \frac{2^{\Omega(\log_{c} h)}}{O(\log^2 h)} \\
    & = \frac{2^{\Omega(\sqrt{\log h})}}{O(\log^2 h)} \\
    % & \text{(recall that $c\le h^{\epsilon}$)} \\
    & \ge 2^{\Omega(\sqrt{\log h})} . \qedhere
  \end{align*}
\end{proof}


\begin{proof}[Proof of \cref{main_thm}]
  Let $n:=|V(G_h)|=2^{h+1}-1$.  Suppose that $G_h\subsetcong H\boxtimes P\boxtimes K_c$ for some graph $H$ of treewidth $t$ and maximum degree $\Delta$, some path $P$ and some integer $c$.  Then, by \cref{dingy,partitions_vs_products}, $G\subsetcong T\boxtimes P\boxtimes K_{24t\Delta c}$ for some tree $T$.  By \cref{partitions_vs_products}, $G$ has a $T$-partition $\mathcal{T}:=\{B_x:x\in V(T)\}$ and a path-partition $\mathcal{P}:=\{P_y:y\in V(P)\}$ such that $|B_x\cap P_y|\le 24t\Delta c$ for each $(x,y)\in V(T)\times V(P)$.  By \cref{big_lemma}, there exists some $(x,y)\in V(T)\times V(P)$ such that $|B_x\cap P_y| \ge \delta 2^{\log h}$.  Combining these latter two equations implies that $t\Delta c\ge 2^{\Omega(\sqrt{\log h})} = 2^{\Omega(\sqrt{\log\log n})}$.
\end{proof}



\section{Open Problems}

We know that every planar graph $G$ is contained in a product of the form $H\boxtimes P\boxtimes K_3$ where $\tw(H)\le 3$ \cite{dujmovic.joret.ea:planar}. \cref{main_thm} states that, for any $c$, there exists a planar graph that is not contained in any product of the form $T\boxtimes P\boxtimes K_c$ where $T$ is a tree and $P$ is a path.  This leaves the following open problems:

\begin{enumerate}
  \item For each $\Delta\in\{3,4\}$, is every planar graph $G$ of maximum degree $\Delta$ contained in a product of the form $T\boxtimes P\boxtimes K_c$ where $T$ is a tree, $P$ is a path, and $c$ is an absolute constant?

  \item Is every planar graph $G$ of maximum degree $\Delta$ contained in a product of the form $H\boxtimes P\boxtimes K_c$ where the treewidth of $H$ is $2$, $P$ is path, and $c$ is some function of $\Delta$?
\end{enumerate}

\cref{two_tree} shows that $G_h$ is a subgraph of $H\boxtimes P$ where $H$ has treewidth $2$ (and is even outerplanar) and $P$ is a path. Our proof breaks down in this case because, unlike tree-partitions, outerplanar partitions do not satisfy \cref{tree_thingy}.


\begin{figure}
  \begin{center}
    \begin{tabular}{cc}
      \includegraphics[scale=.7]{figs/g_5-2} & \includegraphics[scale=.7]{figs/g_5-3}
    \end{tabular}
  \end{center}
  \caption{An outerplanar partition $\mathcal{H}$ and a path-partition $\mathcal{P}$ of $G_5$ for which $|B\cap P|\le 1$ for each $B\in\mathcal{H}$ and $P\in\mathcal{P}$.}
  \label{two_tree}
\end{figure}

\todo[inline]{DW asks why arxiv identifiers appear twice. Each paper has standard bibliographic information followed by one or more hyperlinks.}

\bibliographystyle{plainurlnat}
\bibliography{tp}

\end{document}
